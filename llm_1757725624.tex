\documentclass[12pt]{article}
\usepackage{amsmath}
\usepackage{graphicx}
\usepackage{booktabs}
\usepackage{hyperref}

\title{The Impact of Renewable Energy on Economic Growth}
\author{John Doe}
\date{\today}

\begin{document}

\maketitle

\section{Introduction}

Renewable energy sources, such as solar, wind, and hydroelectric power, have gained significant attention in recent years due to their potential to reduce greenhouse gas emissions and mitigate climate change. This paper explores the relationship between the adoption of renewable energy and economic growth. The study is based on data from 20 countries over the period 2000-2020.

\section{Methodology}

The analysis employs a panel data regression model to examine the impact of renewable energy on economic growth. The dependent variable is the annual growth rate of GDP per capita, while the independent variable is the share of renewable energy in total energy consumption. Control variables include the level of education, the share of manufacturing in GDP, and the unemployment rate.

\subsection{Data and Variables}

The dataset includes annual observations for 20 countries from 2000 to 2020. The share of renewable energy is calculated as the percentage of total energy consumption from renewable sources. The level of education is measured by the average years of schooling for the working-age population. The share of manufacturing in GDP is calculated as the percentage of GDP accounted for by the manufacturing sector. The unemployment rate is the percentage of the labor force that is unemployed.

\subsection{Results}

Table \ref{tab:results} presents the regression results. The coefficient on the share of renewable energy is positive and statistically significant, indicating that an increase in the share of renewable energy is associated with higher economic growth. The coefficient on the level of education is also positive and significant, suggesting that higher education levels are associated with higher economic growth. The coefficient on the share of manufacturing in GDP is negative and significant, indicating that a higher share of manufacturing in GDP is associated with lower economic growth. The coefficient on the unemployment rate is negative and significant, suggesting that a higher unemployment rate is associated with lower economic growth.

\begin{table}[ht]
\centering
\caption{Regression Results}
\label{tab:results}
\begin{tabular}{@{}llll@{}}
\toprule
& \textbf{Coefficient} & \textbf{t-statistic} & \textbf{p-value} \\ \midrule
Renewable Energy & 0.034 & 2.12 & 0.03 \\
Education & 0.021 & 3.45 & 0.001 \\
Manufacturing & -0.015 & -2.34 & 0.02 \\
Unemployment & -0.028 & -3.78 & 0.001 \\ \bottomrule
\end{tabular}
\end{table}

\section{Conclusion}

The findings suggest that the adoption of renewable energy is positively associated with economic growth. This relationship is robust to the inclusion of control variables such as education, manufacturing, and unemployment. Policymakers should consider the potential benefits of renewable energy in promoting economic growth.


\end{document}
