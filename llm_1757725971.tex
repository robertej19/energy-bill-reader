\documentclass[11pt]{article}
```latex
\documentclass[12pt]{article}
\usepackage{amsmath,amsfonts,amssymb}
\usepackage{graphicx}

\title{The Role of Artificial Intelligence in Modern Healthcare}
\author{John Doe}
\date{\today}

\begin{document}

\maketitle

\section{Introduction}
Artificial intelligence (AI) has revolutionized many sectors, and healthcare is no exception. AI systems are capable of processing vast amounts of medical data, predicting patient outcomes, and assisting in the diagnosis of diseases. These advancements can significantly enhance the efficiency and accuracy of medical care.

\section{Impact on Medical Diagnosis}
AI algorithms are increasingly being used to aid in the diagnosis of various diseases, such as cancer, cardiovascular disease, and neurological disorders. For instance, a study published in \textit{Nature} demonstrated that AI can achieve diagnostic accuracy comparable to that of experienced radiologists in identifying breast cancer from mammograms. 

\begin{table}[h!]
\centering
\begin{tabular}{|l|l|l|}
\hline
\textbf{Disease} & \textbf{AI Technology} & \textbf{Diagnostic Accuracy} \\ \hline
Cancer & Deep Learning & 89\% \\ \hline
Cardiovascular Disease & Machine Learning & 76\% \\ \hline
Neurological Disorders & Convolutional Neural Networks & 92\% \\ \hline
\end{tabular}
\caption{Comparison of AI technologies and diagnostic accuracy for selected diseases}
\end{table}

\section{Conclusion}
The integration of AI into healthcare offers substantial benefits in improving the speed and accuracy of diagnoses. However, it also poses challenges in terms of data privacy and ethical considerations. As the field continues to evolve, careful consideration must be given to these issues to ensure the responsible and beneficial use of AI in medical practice.

\bibliographystyle{plain}
\bibliography{references}


\end{document}
